%% Introduction Section %%
\section{ \textbf{Introduction}}

Today's supercomputer landscape is in flux.
%
Supercomputer architectures are undergoing more extreme changes than they have ever undergone in the past 20 years.
%
One big driving factor for this change are the concerns about power usage as we scale to larger and larger machines.
%
Modern machines are pushing up against a hard power limit meaning that in order to increase performance they must become more power efficient.
%
Traditionally, FLOPS/Watt is used to describe the relationship between power and performance.
%
Architectures are transitioning from fast and complex multi-core CPUs to much larger numbers of slower and simpler processors.
%
The amount of parallelism available on any given node in a supercomputer is growing by factors of hundreds or thousands because of this change.
%
This transition to many-core computing brings new and interesting challenges that need to be overcome.
%

%
In addition to the increase in node-level parallelism, it is unclear which many-core architecture choice will prove to be a winning design.
%
There are many different architectures to choose from when designing a supercomputer, and there is no obviously choice.
%
NVIDIA provides General Purpose Graphics Processing Units (GPGPUs) which are highly parallel throughput-optimized devices.
%
Intel provides Many Integrated Core (MIC) co-processors which provide large vector lanes and many threads.
%
Another option is Field Programmable Gate Arrays (FPGAs) which provide programmable logic circuits.
%
Across the Department of Energy (DOE) National Labs both the NVIDIA and Intel technologies are being pursued in their newest procurements ~\cite{coralWeb, trinityWeb}.
%

Application developers now face a complex and unclear path forward.
%
There are additional levels of complexity and potentially large application changes in order to effectively utilize this increase in parallelism.
%
In addition, an application developer cannot simply port to a new hardware architecture.
%
Instead, the application developer must address the issue of portability as well as performance of the algorithm or risk becoming outdated or unusable very quickly.
%
This problem is especially challenging when optimizations are architecture specific
%

%
Currently, there are a large number of physics and multi-physics applications that must understand how to navigate this complex and challenging landscape.
%
Simply porting to new architectures does not guarantee performance, and will still require applications to be ported individually to multiple architectures.
%
This paper will explore today's current efforts for portable performance solutions in the scope of one physics application, Monte Carlo particle transport.
%

%%%%%%%%%%%%%%%%%%%%%%%%%%%%%%%%%%%%%%%%%%%%%%%%%%%%%%%%%%%%%%%%%%%%%%%%%%%%%%%%%%%%%%