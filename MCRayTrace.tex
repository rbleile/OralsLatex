\subsection{\textbf{Monte Carlo and Ray Tracking}}

One important and often computationally expensive aspect of Monte Carlo transport is the step that determines if the particle will collide with any background geometry, or at least cross into a zone of a different material.
%
This is done through a similar method as that used in ray tracing.
%
Ray tracing is a technique in computer graphics for ``generating an image by tracing the path of light through pixels in an image plane and simulating the effects of its encounters with virtual objects"~\cite{wikipediaRayTracing}.

The general process of ray tracing is very similar to Monte Carlo transport in the need to do many intersection tests from potentially scattered sources.
%
Bergmann decided to study the potential of using the power of a highly optimized GPU ray tracing library, OptiX~\cite{2014development}.
%
OptiX is a scalable framework for building ray tracing applications~\cite{optixProgrammingGuide}~\cite{parker2010optix}.

%
The first study conducted was to determine the optimum configuration for OptiX as well as the capability for OptiX to be initialized with random starting points and directions as is most likely to be the case in a Monte Carlo application.
%
When using a ray tracing library it is important to consider the two areas that can scale: the number of concurrently traced rays and the number of geometrical objects in the scene.
%
Nuclear reactor simulations might contain thousands of material zones in complex geometric layouts; knowing this last scaling parameter is especially important to not overlook~\cite{2014development}.
%
In these studies, the rates became fairly consistent after reaching $10^6$ particles.
 %
Bergmann also notes some important points, such as which acceleration structure was always best and when memory become a constraint on the problem that could be run.
 %
 The conclusion from this study was that OptiX could be used to handle the geometry representation in a Monte Carlo neutron transport code.
 %
 Additionally, for best performance one should use a primitive-based geometry instancing method, a BVH acceleration structure, and run as many parallel rays as possible.
%

%
In addition to the use of a pre-existing tool like NVIDIA's OptiX library, other groups looked at optimizing Monte Carlo transport by focusing on treating it like a ray tracing problem.
%
Xiao et al.~\cite{xiao2015monte} focused on the data locality issues in all ray tracing applications on GPUs.
%
They describe a new data locality method based on task partitioning and scheduling in order to enhance spatial and temporal data locality by ordering random rays into coherent groups.
%
By applying this method they achieved a 6-8X speedup over the previous GPU version of radiation therapy Monte Carlo transport. 
%
Despres et al.~\cite{despres2008stream} studied the ray tracing algorithm for tracing a path through a grid in the context of Monte Carlo applications.
%
Their GPU implementation of the Suddon algorithm, showed a speedup factor of 6X over the CPU.
%
This work provides context for an important portion of the Monte Carlo transport problem, a look at the transport piece itself.
%

%
These examples show that progress in connected fields can positively impact the approaches in Monte Carlo transport.
%
Ray tracing is only one aspect of a full Monte Carlo transport application but as it can be greatly beneficial to look at work done in these related fields and bring those ideas back into the full application.

