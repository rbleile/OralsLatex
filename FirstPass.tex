\subsection{ \textbf{First Pass at GPU Computing}}

This section will analyze the different approaches that people have taken to get their Monte Carlo transport codes working for GPU architectures.
%
It will begin by comparing and contrasting different approaches by evaluating a few key areas of the studies that have been done: accuracy, performance and algorithmic choices.
%
Following will be an evaluation of the effectiveness of the approaches for the range of problems being addressed.
%
As a side note it is important to notice that I will be reporting speedups as reported by each paper on the hardware they were using at the time of their study; the GPU hardware has changed in computing power dramatically over the last ten years in terms of performance and additional features.

\subsection*{ \textbf{Accuracy:} }

One of the first considerations the scientific community has when being introduced to a new computing platform is what levels of accuracy can they achieve with their simulation codes.
%
Since the change from CPU to GPU computing brings a completely different hardware design it is important to understand how that design might affect the accuracy of any calculations it is performing.
%
This concern was especially important in the early days of GPU computing when double precision was not supported and often even single precision answers would provide slightly different results.
%
There are three key areas of accuracy that we will look at with these studies, being: Floating point precision, differences between CPU and GPU results, and IEE-754 compliance.
%

%
It was common that in early GPU computing, when double floating point precision was not supported or supported well, that people started thinking about GPU computing as not being accurate enough for their needs.
%
Many early attempts at GPU computing includes discussions of accuracy in order to validate the correctness of their results.
%
While modern GPGPUs support double precision much better than before making much of the worry irrelevant, it is still important to consider the accuracy of a method that runs on a new hardware and may use a new algorithm.
%

\subsubsection*{\textbf{Floating Point Accuracy}}

One of the primary concerns of the early GPU studies involved understanding the limits of floating point arithmetic on the GPU architecture.
%
Nelson in his thesis work~\cite{nelson2009monte} describes one of his primary accuracy considerations as being the difference between single and double precision calculations.
%
In older GPU hardware there was no support for double precision in the hardware and so in order to achieve double precision significantly more calculations were needed.
%
In modern GPU hardware 64 bit double precision is becoming increasingly better supported and in the GPGPU cards there are dedicated double precision units and all of the necessary hardware changes required to include them.
%

\subsubsection*{\textbf{Differences Between CPU and GPU results}}

Of larger concern than the differences between single and double precision, are differences in results that arise when using the same precision.
%
This concern can be explained by understanding how floating-point math is accomplished on a computer~\cite{goldberg1991every}.
%
There are two main reasons that differences arise.
%
The first is that floating point mathematical operations that are done in a different order might produce a different result and due to the nature of parallel computing often you cannot know or guarantee the order a set of calculations will be performed in.
%
The second reason is that modern day CPUs using x86 processors perform math internally on 80bit registers while a GPU does it on 32 bit (single precision) or 64 bit (double precision) registers.
%
Because of this, each math operation on a CPU might stay in registers and only be rounded down to 64 bits when it is saved to memory.
%

%
Jia et. al ~\cite{jia2010development} showed that in their development of a Mont Carlo dose calculation code they could achieve speedups of 5 to 6.6 times their CPU version while maintaining within 1\% of the dosing for more than 98\% of the calculation points.
%
They considered this adequate accuracy to consider using GPUs for doing these computations.
%
Yepes et. al~\cite{yepes2010gpu} also considered accuracy in their assessment of their GPU implementation.
%
They concluded that in term of accuracy there was a good agreement between the dose distributions calculated with each version they ran, the largest discrepancies being only $\sim$3\%, and so they could run the GPU version as accurately as any general-purpose Monte Carlo program.
%
As these two groups have shown this amount of error is often very small and over the entire course of the simulation only brings 1-3\% errors.

\subsubsection*{ \textbf{IEE-754 Compliance} }
Nelson discussed accuracy in his thesis work~\cite{nelson2009monte}, stating that during the time of his work the floating-point arithmetic accuracy was not fully IEEE-754 compliant which opens the question of accuracy without a more fully featured test. 
%
Additionally, since NVIDIA has complete control over the implementation of floating point calculations on their GPUs there may be differences between generations that mitigate the usefulness of an accuracy study on one generation of hardware.
%
Current generations of the NVIDIA GPU hardware are IEE-754 compliant however. 
%
In order to address issues of floating point accuracy they have even included a detailed description of the standard and the way CUDA follows the standard showing that at least while floating point accuracy is still a concern it is no more a concern than it was on a CPU implementation.~\cite{cudaToolkitv7.5}
%

\subsection*{\textbf{Performance:}}

A second factor that is important to people making their first pass at GPU Monte Carlo is performance.
%
Most early GPU studies emphasize the speedups between CPU and GPU as the primary advantage for moving over to the GPU hardware.
%
Given the change in supercomputing designs these comparisons have become increasingly more important.
%

%
Often, performance is compared to the hardware maximums such as peak of FLOPS or Memory Bandwidth.
%
It is often assumed that an increase in available FLOPS will translate directly into incredible performance gains.
%
In Lee et al.'s Debunking the 100X GPU vs. CPU myth~\cite{lee2010debunking}, this discussion of performance is brought into new light showing the relative performance gains for different types of applications.
%
The important thing to consider is the limiting factor between the hardware and the code.
%
Because of this comparing current performance with that of peak performance is often very misleading.
%

%
The following discussions show the relative performances of Monte Carlo transport applications that underwent their initial transformations or studies to use the GPU hardware.
%
We will not see the 100x performance that is often sought after, but instead we can understand the impact that each applications problem, algorithms, and implementation differences had on the performance as a whole.
%

\subsubsection*{\textbf{Photon Transport}}
%
Badal and Badano~\cite{badal2009accelerating} present work on photon transport in a voxelized geometry showing results around 27X over a single core CPU.
%
Their work emphasizes simply using GPUs instead of CPUs and the advantage as GPUs continue to increase in performance faster than CPUs.
%

\subsubsection*{\textbf{Neutron Transport}}
%
Nelsons work presented in his thesis~\cite{nelson2009monte} shows a variety of models and considerations for his performance results.
%
His work solving neutron transport considered multiple models for running the problem and optimizing for the GPU.
%
The model that produced his best results shows 19.37X from a 49,152 neutrons per batch run for single precision.
%
The same model shows 23.91X when using single precision and fast math,
%
For double precision performance the model labeled model four had the fastest speedups with 11.21X and 12.66X with fast math.
%

\subsubsection*{\textbf{Gamma Ray Transport}}
%
Work presented by Tickner~\cite{tickner2010monte} on X-ray and gamma ray transport uses a slightly modified scheme from the others by launching particles on a per block basis.
%
In this way he hoped to remove the instruction level dependancies between particles running on the GPU hardware.
%
In this work he showed he was capable of producing speedups of up to 35X over a single core cpu, and a significant improvement on any per-thread methods we have seen so far.
%

\subsubsection*{\textbf{Coupled Electron Photon Transport}}
%
Jia et. al's  work~\cite{jia2010development} in a dose calculation code for coupled electron photon transport follows a relatively straight forward algorithm.
%
In their work they offload the data and computations to the GPU, simulate the particles, and then copy memory back.
%First they copy the necessary data to the GPU, then they launch with each thread independently computing the necessary work for their particle.
%%
%Finally, when the correct number of particles has been simulated the results are brought back to the CPU and the program terminates.
%%
This method produced a modest performance increase on a GPU of around 5 to 6.6X over their runs on a CPU.
%
The limitation of this speedup was attributed to the branching of the code and that effect it had on the GPU hardware.

\subsubsection*{\textbf{Track Repeating Alogorithm}}
In contrast to Jia et al's work Yepes et al~\cite{yepes2010gpu} showed that a different algorithm could greatly improve results.
%
By converting a track-repeating algorithm instead of a full Monte Carlo, Yepes et al. gained aroung 75X the performance on the GPU over the CPU.
%
It is thought that the simpler logic of this algorithm generated threads which followed closer logic to that of the algorithm presented in Jia at al's work.

\subsubsection*{\textbf{Performance Evaluation}}
Throughout all of these examples one common theme can be seen.
%
Performance can be gained doing Monte Carlo on the GPU.
%
Performance can be more difficult to get due to the highly divergent nature of the full Monte Carlo application.
%
Methods to deal with this divergence can show promising results that are worthy of further study.
%
These outcomes are expected outcomes since Monte Carlo applications are embarrassingly parallel ( good for GPUs ) but also incredibly divergent ( bad for GPUs ).
%

In this section we see a wide range in performances, from as low as 5x to as high as 75x.
%
While simplifications played a large role in the 75x algorithm we do see a full monte carlo application achieving speeds of 35x in the case of the work by Tickner~\cite{tickner2010monte}.
%
I think that is important to note that while some of the differences in performance are due to the nature of each problem being solved, the algorithmic choices made can have a significant impact on the GPU implementations.
%

\subsection*{Algorithms:}

Based on the performance studies we have just seen, it is important to highlight the algorithmic approaches that were taken so that we can understand the performances of each approach.
%
If we can clearly find algorithms that show positive performance results than other codes can implement them for potential gain.
%
In this section we are going to look closely at a few of the important or interesting algorithms we have seen attempted.
%

%
Monte Carlo transport applications tend to follow a simple model where each tracked particle is given its own thread and computations progress in an embarrassingly parallel fashion. 
%
On a GPU this also makes sense as a starting point since particles are independent and this progression leads to a simple natural parallel approach.
%
It is often pointed out however that due to the divergent nature of Monte Carlo this approach might not be the best way organize Monte Carlo codes on GPU hardware.
%

\subsubsection*{\textbf{Particle-Per-Block}}
We will first look at an alternative approach, the particle-per-block tracking algorithm described by Tickner~\cite{tickner2010monte}.
%
First each tracked particle or quantum of radiation is given to a block of threads.
%
Then calculations are performed for one particle on each block of threads.
%
For example the particle intersection tests with the background geometry can be performed in parallel on those threads for each piece of geometry that particle might be able to collide with.
%
Areas where these parallel instructions can be utilized within a particles calculation are then used by the threads in a block computing for that particle.
%

This particle-per-block technique has shows promise as an effective way to counteract the divergence issue.
%
Particles often diverge quite quickly from one another in the code paths they follow.
%
This means that threads in a block are not always able to travel in lock step and can cause some serialization of the parallel regions.
%
By using only one particle per block the divergence problem is nearly entirely removed from the equation.
%
Additionally this method introduces new areas of parallelism that are not otherwise being taken advantage of, instruction level parallelism in the calculations for a single particle.
%

%
This method however, does not take full advantage of the parallelism in the hardware like those methods that do not mind the divergence do.
%
Many threads can execute simultaneously at once within a block and only groupings of 32 threads are held in a WARP forced into the lockstep pattern that causes potential slowdowns.
%
By running only one particle per block you are sacrificing some parallelism as not all tasks to calculate a particles path are parallel operations.
%
Additionally, since warps are scheduled out of thread blocks any particle operations that are not done in parallel among the threads of a block are serializing themselves in a similar manner as to those algorithms that run one thread per particle waiting while divergent particles have a turn.
%

In summary I think that this method has some merit if it can find enough parallel work in the thread block to execute additional parallel tasks that would otherwise be stalled if following a simpler method.
%
I also think that this method might end up showing the same characteristics of the simpler particle-per-thread model if the extra parallelism is not found, and instead loose out on the parallelism provided by particles that are not divergent from one another. 

\subsubsection*{\textbf{Event-Based Approaches}}

A second possibly more obvious method to escape the divergence issue is to switch particle tracking algorithms more dramatically from a history based version to an event based version.
%
We will have a discussion of this further in Section~\ref{sec:EventBased} later in this paper.
%
Event based approaches require much more work then simply transforming an existing code to use the history based version on the GPU.
%
And as Du et al discovered in their attempt at a event based Monte Carlo version of the Archer code~\cite{xu2015archer}~\cite{du2013evaluation}~\cite{liu2015comparison}~\cite{su2013monte},  getting any speedups with that method has a whole new host of challenges to overcome.

\subsubsection*{\textbf{Voxelization Approaches}}

This method was used for comparison on the GPUs.
%
Voxelization of a geometry was done for each voxel and this process involved: ray-stabbing numbers counted on the GPU and then a parity-counting method was run on the CPU to detect if the voxel was inside the mesh surface~\cite{na2010deformable}.
%
This method contained no divergence since all threads follow the exact same code paths.
%
This process is often done to voxelize geometries for before Monte Carlo codes can be run.
%
Doing this algorithm with no divergence produces a 45.5x speedup on the GPU over the CPU.
%
This example is in Ding et al.'s evaluation report~\cite{ding2011evaluation} in order to show the performance of the same GPU on different aspects related to Monte Carlo transport.

\subsection*{ \textbf{Evaluation:}}

\begin{table}
\caption { GPU speedup evaluation results } \label{tab:GPUPerfEval} 
\begin{center}
\begin{tabular}{ |C{.10\textwidth}|C{.10\textwidth}|C{.10\textwidth}|C{.10\textwidth}|}
\hline
 & & & \\
Case & Execution Time $T_{CPU}$ (minutes) & Execution Time $T_{GPU}$ (minutes) & Speed-up factor $T_{CPU}/T_{GPU}$ \\
 & & & \\
 \hline
 & & & \\
Neutron Transport Problem & $~ 0.496$ & $ ~0.017$ & $~ 29.2$ \\ 
 & & & \\
\hline
 & & & \\
eigenvalue/criticality problem & $ 4.25 $ & $ ~ 0.5 $ & $ ~ 8.5 $ \\
 & & & \\
\hline
 & & & \\
Voxelization & $ 2380.4 $ & $ ~ 52.3 $ & $~ 45.5 $ \\
 & & & \\
\hline
\end{tabular}
\end{center}
\end{table}

A number of studies were conducted by groups identifying the potential benefits of GPU hardware but also the hardware and software issues when developing Monte Carlo applications.
%
Among these concerns are memory limitations, lack of ECC support, lack of software optimization, limitations of SIMD architecture, clock speeds, and complex memory allocation schemes.
%
In addition the achieved performance was often not more than could be gotten with unchanged codes on a cluster.
%
In some cases though speedups were large and easy to achieve such as the 45X speedup of the voxelized approach. 
%
The results from Ding et al.'s evaluations can be seen in Table~\ref{tab:GPUPerfEval}.
%
The only strong conclusion from these works are that a clear and defined path are not yet known on how to take full advantage of the available parallelism without suffering performance penalties in turn. ~\cite{ding2011evaluation}
