
\documentclass[conference]{IEEEtran}

% *** CITATION PACKAGES ***
%
% cite.sty was written by Donald Arseneau
% V1.6 and later of IEEEtran pre-defines the format of the cite.sty package
% \cite{} output to follow that of the IEEE. Loading the cite package will
% result in citation numbers being automatically sorted and properly
% "compressed/ranged". e.g., [1], [9], [2], [7], [5], [6] without using
% cite.sty will become [1], [2], [5]--[7], [9] using cite.sty. cite.sty's
% \cite will automatically add leading space, if needed. Use cite.sty's
% noadjust option (cite.sty V3.8 and later) if you want to turn this off
% such as if a citation ever needs to be enclosed in parenthesis.
% cite.sty is already installed on most LaTeX systems. Be sure and use
% version 5.0 (2009-03-20) and later if using hyperref.sty.
% The latest version can be obtained at:
% http://www.ctan.org/pkg/cite
% The documentation is contained in the cite.sty file itself.
%
%
%% *** GRAPHICS RELATED PACKAGES ***
%% 
%\usepackage[pdftex]{graphicx}

% *** MATH PACKAGES ***
%

% A popular package from the American Mathematical Society that provides
% many useful and powerful commands for dealing with mathematics.
%
% Note that the amsmath package sets \interdisplaylinepenalty to 10000
% thus preventing page breaks from occurring within multiline equations. Use:
% after loading amsmath to restore such page breaks as IEEEtran.cls normally
% does. amsmath.sty is already installed on most LaTeX systems. The latest
% version and documentation can be obtained at:
% http://www.ctan.org/pkg/amsmath

% *** SPECIALIZED LIST PACKAGES ***
%
% algorithmic.sty was written by Peter Williams and Rogerio Brito.
% This package provides an algorithmic environment fo describing algorithms.
% You can use the algorithmic environment in-text or within a figure
% environment to provide for a floating algorithm. Do NOT use the algorithm
% floating environment provided by algorithm.sty (by the same authors) or
% algorithm2e.sty (by Christophe Fiorio) as the IEEE does not use dedicated
% algorithm float types and packages that provide these will not provide
% correct IEEE style captions. The latest version and documentation of
% algorithmic.sty can be obtained at:
% http://www.ctan.org/pkg/algorithms
% Also of interest may be the (relatively newer and more customizable)
% algorithmicx.sty package by Szasz Janos:
% http://www.ctan.org/pkg/algorithmicx


% *** ALIGNMENT PACKAGES ***
%
%\usepackage{array}
% Frank Mittelbach's and David Carlisle's array.sty patches and improves
% the standard LaTeX2e array and tabular environments to provide better
% appearance and additional user controls. As the default LaTeX2e table
% generation code is lacking to the point of almost being broken with
% respect to the quality of the end results, all users are strongly
% advised to use an enhanced (at the very least that provided by array.sty)
% set of table tools. array.sty is already installed on most systems. The
% latest version and documentation can be obtained at:
% http://www.ctan.org/pkg/array



% IEEEtran contains the IEEEeqnarray family of commands that can be used to
% generate multiline equations as well as matrices, tables, etc., of high
% quality.



% *** SUBFIGURE PACKAGES ***
%\ifCLASSOPTIONcompsoc
%  \usepackage[caption=false,font=normalsize,labelfont=sf,textfont=sf]{subfig}
%\else
%  \usepackage[caption=false,font=footnotesize]{subfig}
%\fi
% subfig.sty, written by Steven Douglas Cochran, is the modern replacement
% for subfigure.sty, the latter of which is no longer maintained and is
% incompatible with some LaTeX packages including fixltx2e. However,
% subfig.sty requires and automatically loads Axel Sommerfeldt's caption.sty
% which will override IEEEtran.cls' handling of captions and this will result
% in non-IEEE style figure/table captions. To prevent this problem, be sure
% and invoke subfig.sty's "caption=false" package option (available since
% subfig.sty version 1.3, 2005/06/28) as this is will preserve IEEEtran.cls
% handling of captions.
% Note that the Computer Society format requires a larger sans serif font
% than the serif footnote size font used in traditional IEEE formatting
% and thus the need to invoke different subfig.sty package options depending
% on whether compsoc mode has been enabled.
%
% The latest version and documentation of subfig.sty can be obtained at:
% http://www.ctan.org/pkg/subfig


% *** FLOAT PACKAGES ***
%
%\usepackage{fixltx2e}
% fixltx2e, the successor to the earlier fix2col.sty, was written by
% Frank Mittelbach and David Carlisle. This package corrects a few problems
% in the LaTeX2e kernel, the most notable of which is that in current
% LaTeX2e releases, the ordering of single and double column floats is not
% guaranteed to be preserved. Thus, an unpatched LaTeX2e can allow a
% single column figure to be placed prior to an earlier double column
% figure.
% Be aware that LaTeX2e kernels dated 2015 and later have fixltx2e.sty's
% corrections already built into the system in which case a warning will
% be issued if an attempt is made to load fixltx2e.sty as it is no longer
% needed.
% The latest version and documentation can be found at:
% http://www.ctan.org/pkg/fixltx2e


%\usepackage{stfloats}
% stfloats.sty was written by Sigitas Tolusis. This package gives LaTeX2e
% the ability to do double column floats at the bottom of the page as well
% as the top. (e.g., "\begin{figure*}[!b]" is not normally possible in
% LaTeX2e). It also provides a command:
%\fnbelowfloat
% to enable the placement of footnotes below bottom floats (the standard
% LaTeX2e kernel puts them above bottom floats). This is an invasive package
% which rewrites many portions of the LaTeX2e float routines. It may not work
% with other packages that modify the LaTeX2e float routines. The latest
% version and documentation can be obtained at:
% http://www.ctan.org/pkg/stfloats
% Do not use the stfloats baselinefloat ability as the IEEE does not allow
% \baselineskip to stretch. Authors submitting work to the IEEE should note
% that the IEEE rarely uses double column equations and that authors should try
% to avoid such use. Do not be tempted to use the cuted.sty or midfloat.sty
% packages (also by Sigitas Tolusis) as the IEEE does not format its papers in
% such ways.
% Do not attempt to use stfloats with fixltx2e as they are incompatible.
% Instead, use Morten Hogholm'a dblfloatfix which combines the features
% of both fixltx2e and stfloats:
%
% \usepackage{dblfloatfix}
% The latest version can be found at:
% http://www.ctan.org/pkg/dblfloatfix


% *** PDF, URL AND HYPERLINK PACKAGES ***
%\usepackage{url}
% url.sty was written by Donald Arseneau. It provides better support for
% handling and breaking URLs. url.sty is already installed on most LaTeX
% systems. The latest version and documentation can be obtained at:
% http://www.ctan.org/pkg/url
% Basically, \url{my_url_here}.


% *** Do not adjust lengths that control margins, column widths, etc. ***
% *** Do not use packages that alter fonts (such as pslatex).         ***
% There should be no need to do such things with IEEEtran.cls V1.6 and later.
% (Unless specifically asked to do so by the journal or conference you plan
% to submit to, of course. )

\usepackage{cite}
\usepackage{amsmath}
\interdisplaylinepenalty=2500
\usepackage[vlined,linesnumbered,algoruled]{algorithm2e}
\usepackage{multirow}
\usepackage{graphicx}

% correct bad hyphenation here
\hyphenation{op-tical net-works semi-conduc-tor}

\usepackage{csquotes}

\begin{document}

\title{The State of Monte Carlo Neutron Transport: \\
The Role of GPUs and Portable Performance Abstractions }

\author{\IEEEauthorblockN{Ryan Bleile}
\IEEEauthorblockA{School of Computer and Information Science\\
University of Oregon\\
Eugene, Oregon 97403\\
Email: rbleile@cs.uoregon.edu}
}

% make the title area
\maketitle

% As a general rule, do not put math, special symbols or citations
% in the abstract
\begin{abstract}
Since near the beginning of electronic computing Monte Carlo neutron transport has been a fundamental approach to solving nuclear physics problems.
%
Over the past few decades Monte Carlo transport applications have seen significant increase in their capabilities and decreases in time to solution.
%
Research efforts have been focused on areas such as MPI scalability, load balance with domain decomposition, and variance reduction techniques.
%
In the last few years however the landscape has been changing.
%
Due to the inherently parallel nature of these applications Monte Carlo transport applications are often used in the super-computing environment.
%
Supercomputers are changing, becoming increasingly more parallel on node with GPGPUs and/or Xeon Phi co-processors powering the bulk of the compute capabilities.
%
In order to fully utilize the new machines capabilities it is becoming ever more important to migrate to a many-core perspective of any computing algorithms.
%
Monte Carlo transport applications, like many others, have the potentially difficult task of figuring out how to effectively utilize this new hardware.
%
Many groups have taken the initial steps to look into this problem or have focused their efforts on a sub-problem, such as continuous energy lookups.
%
In other fields a promising approach is displayed in the use of portable performance abstractions in order to specify what is parallel.
%
These abstractions provide the foundations for applications to write code once and run on any supported platform.
%
This paper describes the state of the art in Monte Carlo neutron transport, with a special emphasis on the affects of upcoming architectures.
%
\end{abstract}

\tableofcontents

% no keywords

% For peer review papers, you can put extra information on the cover
% page as needed:
% \ifCLASSOPTIONpeerreview
% \begin{center} \bfseries EDICS Category: 3-BBND \end{center}
% \fi
%
% For peerreview papers, this IEEEtran command inserts a page break and
% creates the second title. It will be ignored for other modes.
\IEEEpeerreviewmaketitle

%% Introduction Section %%
\section{ \textbf{Introduction}}

Today's Supercomputer landscape is in flux.
%
Supercomputer architectures are making more extreme changes then they have undergone in 20 years.
%
One big driving factor for this change are the concerns about energy usage as we scale to larger and larger machines.
%
A metric, FLOPS/Watt is often used to describe this relationship.
%
In order to maximize the FLOPS/Watt metric, architectures are shifting away from fast and complex multi-core CPUs and adding in much larger numbers of much slower simpler processors.
%
The amount of parallelism available on any given node in a supercomputer is growing but factors of hundreds or thousands because of this change.
%
This change brings new and interesting challenges that need to be overcome.
%

%
In addition to the increase in node level parallelism, it is unclear which architecture choice will prove to be a winning design.
%
Currently there are many different architectures to choose from when designing a supercomputer and there is no obvious choice to place one design above the others, or if any of these designs are going to end up above the others.
%
NVIDIA provides General Purpose Graphics Processing Units (GPGPUs) which are a highly parallel throughput optimized devices.
%
Intel provides their Many Integrated Core (MIC) co-processor which provides large vector lanes and many threads.
%
Other groups are turning to Field Programmable Gate Arrays (FPGAs) for a solution.
%
Across the Department of Energy (DOE) National Labs both the NVIDIA and Intel approaches are being pursued in their newest procurements ~\cite{coralWeb, trinityWeb}.
%

Application developers now face a complex and unclear path forward.
%
There is additional levels of complexity and potentially large application changes that will need to be made in order to effectively utilize this increase in parallelism.
%
In addition, an application programmer cannot simply begin a cycle of porting to a new hardware architecture.
%
Instead, applications need to address the issue of portability as well as performance or they run the risk of becoming outdated or unusable very quickly.
%
This problem is especially challenging when optimization choices for one architecture can contradict with optimization choices on another architecture.
%

%
There are a large number of physics and multi-physics applications that exist today that must figure out how to navigate this complex and challenging landscape.
%
Simple ports to new architectures are often not enough to guarantee performance, and will still require applications to be ported multiple times to multiple architectures.
%
This paper will explore these concerns and today's current efforts for portable performance solutions in the scope of one of these physics applications, Monte Carlo particle transport.
%

%%%%%%%%%%%%%%%%%%%%%%%%%%%%%%%%%%%%%%%%%%%%%%%%%%%%%%%%%%%%%%%%%%%%%%%%%%%%%%%%%%%%%%

\section{ \textbf{What is Monte Carlo Particle Transport?}}

\begin{displayquote}
"The first thoughts and attempts I made to practice [the Monte Carlo method] were suggested by a question which occurred to me in 1946 as I was convalescing from an illness and playing solitaire. The question was what are the chances that a Canfield solitaire laid out with 52 cards will come out successfully? After spending a lot of time trying to estimate them by pure combinatorial calculations, I wondered whether a more practical method than "abstract thinking" might not be to lay it out say one hundred times and simply observe and count the number of successful plays. This was already possible to envisage with the beginning of the new era of fast computers, and I immediately thought of problems of neutron diffusion and other questions of mathematical physics, and more generally how to change processes described by certain differential equations into an equivalent form interpretable as a succession of random operations. Later... [in 1946, I ] described the idea to John von Neumann and we began to plan actual calculations."
\end{displayquote}
\begin{displayquote}
- Stan Ulam 1983 ~\cite{theMCM}
\end{displayquote}

John von Neumann became interested in Stan Ulam's idea and outlined how to solve the neutron diffusion and multiplication problems in fission devices.
%
Since this time Monte Carlo methods have continued to be a primary way for solving many questions in neutron transport~\cite{theMCM}.
%

\subsection{ \textbf{ Definition} }

In Computational Methods of Neuron Transport~\cite{LewisCMNT}, Lexis and Miller describe Monte Carlo transport as a simulation of some number of particle histories by using a random number generator.
%
For each particle history that is calculated, random numbers are generated and used to sample probability distributions describing the different physical events a particle can undergo, such as scattering angles or the length between collisions.
%
Ivan Lux and Laslo Koblinger further the previous definition in their book Monte Carlo Particle Transport Methods: Neutron and Photon Calculations:
%
\begin{displayquote}
"In all applications of the Monte Carlo method a stochastic model is constructed in which the expected value of a certain random variable (or of a combination of several variables) is equivalent to the value of a physical quantity to be determined. This expectation value is then estimated by the averaging of several independent samples representing the random variable introduced above. For the construction of the series of independent samples, random numbers following the distributions of the variable to be estimated are used." ~\cite{LuxMCPTM}
\end{displayquote}

% 
Estimating a quantity takes on the following mathematical form:
$$ \hat{x} = \frac{1}{N} \sum_{n=1}^{N} x_{n}, $$
where $x_{n}$ represents the contribution of the $n$th history for that quantity.
%
For the Monte Carlo method, we tally the $x_{n}$ from each particle history in order to compute the expected value $\hat{x}$~\cite{LewisCMNT}.
%

%
One very important question is how the estimated value $\hat{x}$ compares to the true value $\bar{x}$.
%
It turns out that the uncertainty in $\hat{x}$ decreases with increasing numbers of particle histories, and generally falls off asymptotically proportionate to $N^{-1/2}$~\cite{LewisCMNT}.
%

\subsection{ \textbf{ The Equation} }

%
The equation being solved by the neutron transport problem, shown below, displays each of the pieces that makes up a full Monte Carlo transport algorithm.
%
Large numbers of particles are used to create accurate estimations for each measured quantity.
%
The following equation is known as the Linearized Boltzmann transport equation for neutrons:
%
$$
\frac{1}{\nu} \frac{ \partial \Psi ( \vec{r}, E, \Omega, t ) }{\partial t}\ +\ 
(\nabla \cdot \Omega ) \Psi ( \vec{r}, E, \Omega, t )\ +\ 
\Sigma_a (\vec{r}, E ) \Psi ( \vec{r}, E, \Omega, t )
$$
$$=$$
$$
\int _{E '} \int _{\Omega '} \Sigma_{s} ( \vec{r}, E', \Omega ' \rightarrow E, \Omega ) \Psi (\vec{r}, E, \Omega, t) d \Omega ' dE'\ +\ 
$$
$$
\chi (E) \int _{E'} \nu (E') \Sigma_{f} (\vec{r},E') \int _{\Omega '} \Psi ( \vec{r}, E, \Omega, t ) d \Omega ' dE'\ +\ $$
$$
S_ext(\vec{r}, E, \Omega, t )
$$
%
where $ \Psi ( \vec{r}, E, \Omega, t ) $ is angular flux, 
$\Sigma_a (\vec{r}, E )$ is the macroscopic cross section for particle absorption, 
$\Sigma_{s} ( \vec{r}, E', \Omega ' \rightarrow E, \Omega )$ is the macroscopic cross section for particle scattering,
$\Sigma_{f} (\vec{r},E') $ is the macroscopic cross section for particle production from a fission collision source,
$\chi (E)$ is a secondary particle spectrum from he fission process,
$\nu (E)$ is the average number of particles emitted per fission,
$S_ext(\vec{r}, E, \Omega, t )$ represents an external source,
$\vec{r}$ is the spacial coordinates,
$E$ is the energy,
$\Omega$ is angular direction,
and $t$ is the term for time~\cite{gentileMCPTAPO}.
%

\subsection{ \textbf{ Algorithm Approach} }

There are many ways to solve this problem.
%
The most common method is to track individual particle histories until a predetermined amount of particles has been simulated.
%
This method is known as the history based approach.
%
In order to simulate a particle, the distance the particle must travel before it has any interaction must be computed and compared with the rest.
%
The interaction with the shortest distance is chosen followed by updating the particle and tallies based on the distance traveled and interaction occurring.
% 
Algorithm~\ref{alg:historybased} shows the history based approach for a simple research code~\cite{alpsmc1} that has simple properties we can use to describe the method.
%
Algorithm~\ref{alg:mcmethod} shows the outer most scope of a Monte Carlo Problem for referencing where different optimizations or stages occur.
%
For example, Algorithm~\ref{alg:historybased} takes place inside the Cycle loop of Algorithm~\ref{alg:mcmethod} and shows only the steps for Cycle Tracking.

\begin{algorithm}
\DontPrintSemicolon
\caption{History based Monte Carlo algorithm}
\label{alg:historybased}
\ForEach{particle history}
{ 
    \While{particle not escaped or absorbed}
    {
       sample distance to collision in material\;
       sample distance to material interface\;
       compute distance to cell boundary\;
       select minimum distance, move particle, and perform event\;
       \If{particle escaped spatial domain}
       {
          update leakage tally\;
          end particle history\;
       }
       \If{particle absorbed}
       {
          update absorption tally\;
          end particle history\;
       }
    }
}
\end{algorithm}

\begin{algorithm}
\DontPrintSemicolon
\caption{Monte Carlo Method}
\label{alg:mcmethod}
Parse Inputs\;
\ForEach{Cycle}
{
	Cycle Initialize\;
	Cycle Tracking\;
	Cycle Finalize\;
}
Gather Tallies\;
\end{algorithm}






\section{ \textbf{ State of the Art: Monte Carlo Research} }

There is a long history of research and improvements for Monte Carlo transport problems.
%
Further, understanding this path, and the machines that the approaches were designed for, helps to guide analysis of more recent efforts.
%
In the field of Monte Carlo transport, most research in the last 5 years has been related to one of the two topics: (1) GPGPU computing or (2) physics improvements (as opposed to computer-science based research).
%
Review and discussion of GPU research in Section~\ref{sec:SOTAGPU}.
%
This section looks at non-GPU Monte Carlo transport research, namely parallel performance on CPU architectures, parallel load balancing, optimizations in nuclear data look-ups, and variance reduction research.
%

	\subsection{Parallel Performance}

In this section we will see the parallel performance of a number of different Monte Carlo particle transport applications on different architectures ranging from the vector machines of the 80's to multi-core compute clusters.
%

\subsection*{ \textbf{Shared Memory Performance}} 

Shared memory systems refer to machines or models where all processors can access the same memory space.
%
Taking this a step further the unified memory architecture (UMA) shared memory systems not only do all processors have access to the same memory but they also have access to all memory in  the same time~\cite{el2005advanced}.
%
One type of shared memory system that was popular in the 1970's and 1980's was the vector machine.
%
Vector machines took the shared memory system and added additional synchronicity to the system by making all of the processors issue the same instruction~\cite{russell1978cray}.
%

\subsection*{Vector Machine Performance}

%
In the 1980's Monte Carlo transport algorithms began adapting "event-based" methods in order to vectorize their algorithms for use on a vector machine.
%
These new algorithms were used because the traditional history based approach has complete independence of particle histories.
%
But in order to effectively utilize the vector architecture particles must be computed on the same code paths.
%
By changing the algorithm to follow events instead of histories the Monte Carlo method could be used in a vector based approach. ~\cite{martin1986monte}
%

%
One common element when reviewing the work done in this area is to see that the vector approach is often related to stacks, and properly organizing particles into the right stack so that calculations can be preformed~\cite{brown1984monte, bobrowicz1984vectorized}.
%
Another approach is to try to use only one main stack and pull off only the minimum information needed to compute the events into sub-stacks~\cite{martin1986monte}.
%
With each of these approaches particle events determine how the particles are organized or what information is needed for processing.
%
The main drawback to the event based approach is the added time processing data movement or sorting.
%

%
Brown reported theoretical speedups of 20x-85x for his consideration and deemed this approach well worth the efforts required to change codes around in order to use this approach ~\cite{brown1984monte}.
%
Martin saw speedups randing from 5x to 12x depending on the problem and the machine he was running on by using the single big stack, sub-stack approach~\cite{martin1986monte}.
%
Bobrowicz explicit stack approach reaches speedups of around 8x - 10x compared with the original history-based approach~\cite{bobrowicz1984vectorized}.
%
Finally Burns in using a LANL Benchmark code GAMTEB showed he could acheive similar performance to Bobrowics by following a similar approached as that laid out by Brown~\cite{burns1989vectorization}.
%

\subsection*{Multi-Threaded Architecture Performance}

%
Other shared memory systems, separate from vector machines, where tried in this time.
%
One such machine was the Tera Multi-Threaded Architecture (MTA).
%
This approach focused on the use of incredibly parallel processors, hardware threading, and a simple shared memory, no cache, design.
%
The idea was by focusing on threading they could mask away memory latency~\cite{majumdar2000parallel, snavely1998multi}.
%

%
One Photon transport application tried two methods of parallelizing their application on the Tera MTA. 
%
For their problem the zones and energies of the region needed to be looped over and photons falling in those ranges where then computed.
%
So for their application they chose to parallelize over zones and also over zones and energies at once through loop unrolling.
%
Table~\ref{tab:MTAPerf} shows that the parallelization on the MTA over zones and energies maintains incredible efficiency giving their application good speedups here, while parallelizing over only zones does not expose enough parallel work to hide memory latency and so efficiency drops of quickly.
%
\begin{table}
\caption {Parallel performance on the MTA using multithreaing} \label{tab:MTAPerf} 
\begin{center}
\begin{tabular}{|c|c|c|c|}
\hline
Procs & Time (sec) & Speedup & Efficiency \\
\hline
\multicolumn{4}{|c|}{Parallelization by zones only} \\
\hline
1 & 764 & 1.00 & 1.00 \\
\hline
2 & 400 & 1.91 & 0.95 \\
\hline
4 & 227 & 3.37 & 0.84 \\
\hline
8 & 167 & 4.58 & 0.57 \\
\hline
\multicolumn{4}{|c|}{Parallelization by zones and energies} \\
\hline
1 & 745 & 1.00 & 1.00 \\
\hline
2 & 370 & 2.01 & 1.01 \\
\hline
4 & 187 & 3.98 & 0.99 \\
\hline
8 & 94 & 7.92 & 0.99 \\
\hline
\end{tabular}
\end{center}
\end{table}

More modern systems utilize shared memory ideas as well, with a majority of the scientific efforts utilizing openMP threading models for shared memory processing.
%
Often this model is overlooked in preference of distribued computing via MPI but that is not always the case.
%
Given an all particle method, openMP codes tend to scale incredibly well with the only draw backs having to do with those few areas requiring atomic operations.
%
With, in my experience, an nearly perfect efficiency in the case of no atomic operations and plenty of work.

\subsection*{ \textbf{Distributed Memory Performance}}

One of the major transitions in supercomputing came with the shift from vector computing to distributed memory computing.
%
This type of computing is most often done with MPI and has for the last 20 years and to this day been a primary method of achieving parallel performance on clusters and supercomputers alike.
%

\subsection*{ \textbf{Distributed + Shared Memory Performance}}

	\subsection{ \textbf{ Load Balance and Domain Decomposition} }

In order to achieve high levels of parallelism in transport problems with many geometries or zones, different parallel execution models are used.
%
The two primary models used are domain decomposition and replication.
%
Domains are sections of the problem space or geometry used for organizing where portions of a larger data set are stored. 
%
Domain decomposition involves spatial decomposition of the geometry into domains, and then the assigning of processors to work on specific domains.
%
Replication involves storing the geometry information redundantly on each processor and assigning each processor a different set of particles~\cite{procassini2005load}~\cite{o2005dynamic}.

Load balance of domain replication problems is often simply a trivial splitting of particles across processors.
%
Because of this, load balance is often discussed in conjunction with domain decomposition specifically.
%
Particles often migrate between different regions of a problem, meaning not all spatial domains will require the same amount of computational work.
%
In many applications there is at least one portion of the calculation that must be completed by all processors before all the processors can move forward with the calculation.
%
As a result, if one processor has more work than any other, all of the others must wait for that processor to complete its work~\cite{procassini2005load}~\cite{o2005dynamic}.
%
This load imbalance can cause significant issues with scalability as parallelism is increased from hundreds to millions of processors~\cite{o2013scalable}.
%

\subsection*{When to Load Balance}

%
A key consideration for performing a load balanced calculation is understanding the cost of performing that calculation.
%
If too much time is spent making sure the problem is always perfectly load balanced, then computational resources are being wasted on a non-essential calculation, resulting in overall slower performance.
%
However, if too little resources are devoted to load balancing then the problem will suffer from load imbalance and the negative effects that entails.
%
One solution is to perform load balance at the start of each cycle or iteration of a Monte Carlo transport calculation, but only when that load balance will result in a faster overall calculation~\cite{o2005dynamic}.
%
One algorithm to determine when to load balance is to use a
criterion that can be checked inexpensively each cycle to determine if a load-balance operation should take place~\cite{procassini2005load}~\cite{o2005dynamic}.
%
In these works, the first step is to compute a speedup factor by comparing current parallel efficiency ($ \varepsilon_C $) to what parallel efficiency would be if processors were to redistribute their load ($ \varepsilon_{LB} $).
%
The second step is to predict the run time by using the time to execute the previous cycle ($ \tau_{Phys} $), the speedup factor ($S$), and finally, the time to compute the load balance itself ($ \tau_{LB} $).
%
The final step is to compare the predicted run with and without load balancing to determine if the operation is worthwhile.
%
The equations describing this algorithm then are:

\begin{eqnarray}
S = \frac{\varepsilon_C}{\varepsilon_{LB}} \\
\tau^{'} = \tau_{Phys} \cdot S + \tau_{LB} \\
\tau = \tau_{Phys} \\
if\ (\tau^{'} < 0.9 \cdot \tau \ )\ DynamicLoadBalance()
\end{eqnarray}

\subsection*{Extended Domain Decomposition}

As an extension to the domain decomposition of meshes, O'Brien and Joy demonstrated an algorithm to domain decompose Constructive Solid Geometry (CSG) in a Monte Carlo transport code~\cite{o2009domain}.
%
One key difference between mesh and CSG geometries is that mesh geometries contain a description of cell connectivity, whereas cells defined through CSG do not.
%
In order to domain decompose these CSG cells, each cell was given a bounding box; since each domain is also a box, a test for if a cell belonging inside a domain becomes an axis-aligned box-box intersection test.
%

In addition to pure mesh and pure CSG problems other combinations are sometimes beneficial, such as the combination of mesh and CSG problems where there are large-scale heterogeneous and homogeneous regions.
%
In this method, a mesh region is embedded inside a CSG region allowing for the use of each in whichever region one or the other is more optimal
~\cite{greenman2009enhancements}.

\subsection*{Load Balance at Scale}

%
When load balancing massively parallel computers, 
examining the workload of every processor can affect scalability.
%
O'Brien, Brantley and Joy presented a scalable load balancing algorithm that runs in $\Theta ( log ( N ) )$ by using iterative processor-pair-wise balancing steps that will ultimately lead to a balanced workload.
%
Their algorithm was demonstrated scalability on up to two million processors on the Sequoia supercomputer at Lawrence Livermore National Laboratory~\cite{o2013scalable}.
%

%
The pair-wise load balancing scheme maintained efficiency of 95\% at 2 million processors while the not load balanced runs dropped in efficiency to around 68\% at 2 million processors.
%
In addition, the load-balanced version is able to maintain near perfect scaling up to 2 million processors.
%
By dispersing the workload over processors effectively it also decreases the overall tracking time~\cite{o2013scalable}.
%

Algorithms that interact with particles and geometries are affected when domain decomposition is added.
%
Three algorithms are described in~\cite{o2015particle} which are modified once domain decomposition is added.
%
Specifically a global particle find algorithm, a test for done, and domain neighbor replication.
%
The global particle find is the algorithm used to find where a particle is currently located in the geometry.
%
After domain decomposition a tree search was added to quickly decide which domain a particle is in before then searching in specific geometry elements.
%
The test for done algorithm, which reports if there are any particles left to process, can be easily achieved by using MPI\_I allreduce() in place of a complex hand coded algorithm.
%
Lastly, the domain neighbor replication was found to be an important algorithm to combine with domain decomposition as it increased achieved load balance and reduced the total memory usage.
%



	\subsection{Nuclear Data}

A large part of many Monte Carlo transport calculations is the process of looking up nuclear data information.
%
Both microscopic and macroscopic cross section information is needed in order to understand what reactions a particle undergoing a collision will do.
%
Depending on the problem and the choices made to solve it, time spent looking up nuclear data can often be between 10\% and 85\% of the over all runtime.
%
The problems that spend more time looking up cross section data are often using what is known as the continuous energy model where energy value are stored as a large sequence of points and exact values are found through interpolation.
%
The second method that is used which makes cross section lookups faster but less accurate is multi-group cross sections, where cross section data is stores in some number of bins and all energies that land in the bin are given the same value.
%
This can often reduce the search many orders of magnitude.
%

%
Research that deals with nuclear data lookups is often concerned with speeding up the search for a given cross section at a given energy.
%
This search problem is the main bottleneck in the cross section lookup algorithms.
%
Linear searches, binary searches, and Hash based searches are often employed for this.
%
In addition combining isotopes into a unionized grid is a common method for reducing the total number of searched required, though it greatly increases the memory needed to store the cross section data.
%

Each of the common competing continuous algorithms is well defined and compared by Wang et. al. and are described as follows ~\cite{wang2016competing}:
%
\subsubsection*{ Hashing: } Each material's whole energy range is divided up into N equal intervals, and for every individual isotope inside the material an extra table is established to store isotopic bounding indexes of each interval ~\cite{brown2014new}. The new search intervals are thus largely narrowed with respect to the original range and can be reached by a single float division. The hashing can be performed on a linear or logarithmic scale; the search inside each interval can be performed by a binary search or linear search. In the original paper ~\cite{brown2014new}, a logarithmic hashing was chosen with $ N \simeq 8000 $ as the best compramize between performance and memory usage. Another variant is to perform the hashing at the isotope level.
%
\subsubsection*{ Unionized grid: } A global unionized table gathers all possible energy points in the simulation and seconds table provides their corresponding indexes in each isotope energy grid ~\cite{leppanen2009two}. Every time an energy lookup is performed, only one search is required in the unionized gris and the isotope index are directly provided by the secondary index table. Timing resutls show that this method has a significant speedup over the conventional binary search but can require up to a 36$\times$ more memory space~\cite{a.l.lunda.r.siegel2015}.
%
\subsubsection*{ Fractional cascading: } This is a technique to speedup search operations for the same value in a series of related data sets ~\cite{a.l.lunda.r.siegel2015}. The basic idea is to build a unified grid for the first and second isotopes, then for seconds and third, etc. When using the mapping technique, once we find the energy index in the first energy grid all the following indexes can be read directly from the extra index teables without further computations. Compared to the global unionized methods, the fractional cascading technique greatly reduces memory usage.

%
We will discuss recent work done in the area of nuclear data lookups when we discuss many-core based Monte Carlo research as much of the results are targeted at NVIDA GPGPUs or the Intel XEON Phi many-core coprocessor. 

	\subsection{ \textbf{Variance Reduction Techniques} }

Variance reduction is a key concept in Monte Carlo transport problems.
%
The solutions to Monte Carlo problems are given in the form of statistics and so reducing the variance in those statistics leads to more accurate or easier to compute solutions.
%
Often without some use of variance reduction, certain problems would take an incredible amount of time and computing power to find a solution.
%
The idea behind variance reduction is to increase the efficiency of Monte Carlo calculations and permit the reduction of the sample size in order to achieve a fixed level of accuracy or increase accuracy at a fixed sample size~\cite{kahn1953methods}.
%
Some commonly used variance reduction techniques are common random numbers, antithetic variates, control variates, importance sampling and stratified sampling, although most used in Monte Carlo transport is some form of importance sampling.
%

\subsubsection*{\textbf{Common Random Numbers}} This method of variance reduction involves comparing two or more alternative configurations instead of only a single configuration. Variance reduction is achieved by introducing an element of a positive correlation between the sets. This can be accomplishd through insuring that all configurations of a problem use the same random numbers to find solutions. ``For example, in queueing theory, if we are comparing two different configurations of tellers in a bank, we would want the (random) time of arrival of the Nth customer to be generated using the same draw from a random number stream for both configurations"~\cite{wikipediaVarReduction}.

\subsubsection*{\textbf{Antithetic Variates}} This method of variance reduction involves taking the antithetic path for each path sampled --- so for a given path $ \{ \varepsilon_1, ..., \varepsilon_M \}$ one would also take the path $ \{ -\varepsilon_1, ... , -\varepsilon_M  \} $. This method reduces the number of samples needed and reduces the variance of the sampled paths~\cite{wikipediaAntitheticVaraites}.

\subsubsection*{\textbf{Control Variates}} This method of variance reduction involves creating a correlation coefficient by using information about a known quantity to reduce the error in an unknown quantity. This method is equivalent to solving a least squares system and so is often called regression sampling~\cite{wikipediaControlVaraites}.

\subsubsection*{\textbf{Importance Sampling}} This method of variance reduction involves estimating properties of a particular distribution, while only having samples generated from a different distribution than the distribution of interest. This method emphasizes important values by sampling them more frequently and sampling unimportant values less frequently~\cite{wikipediaImportanceSampling}. This is often achieved through methods known as splitting or Russian roulette. In splitting and Russian roulette particles are each given a weight and if particles enter an area of higher importance they are split into more particles with less weight giving a larger sample size. If particles travel in a region that is not important they undergo Russian roulette where some particles are killed off and others are given more weight to account for those removed~\cite{melnik2000rare}.

\subsubsection*{\textbf{Stratified Sampling}} This method of variance reduction is accomplished by separating members of a population into homogeneous groups before sampling. Sampling each stratum reduces sampling error and can produce weighted means that have less variability than the arithmetic mean of a simple sampling of the population~\cite{wikipediaStratifiedSampling}.

Recent research in the area of variance reduction techniques often includes a specific problem that requires a more focused study to utilize one of these previously described patterns.
%
For example, in the problem of atmospheric radiative transfer modeling, Iwabuchi recently published work describing a proposal for some variance reduction techniques that they can use to help solve the problem of solar radiance calculations.
%
Described are four methods that are developed directly from their problem.
%
The first is to use a type of Russian roulette on values that will contribute small or meaningless amounts to the overall calculation within some threshold.
%
Other methods include approximation methods for sharply peaked regions of the phase space, forcing collisions in under sampled regions, and numerical diffusion to smooth out noise~\cite{iwabuchi2015efficient}.
 

\section{State of the Arr: GPU Research}

a look at gpu research in MC
	\subsection{ \textbf{First Pass at GPU Computing}}

This section will analyze the different approaches that people have taken to get their Monte Carlo transport codes working for GPU architectures.
%
It will begin by comparing and contrasting different approaches by evaluating a few key areas of the studies that have been done: accuracy, performance and algorithmic choices.
%
Following will be an evaluation of the effectiveness of the approaches for the range of problems being addressed.
%
As a side note it is important to notice that I will be reporting speedups as reported by each paper on the hardware they were using at the time of their study.

\subsection*{ \textbf{Accuracy:} }

One of the first considerations the scientific community has when being introduced to a new computing platform is what levels of accuracy can they achieve with their simulation codes.
%
Since the change from CPU to GPU computing brings a completely different hardware design it is important to understand how that design might affect the accuracy of any caluclations it is preforming.
%
This concern was especially important in the early days of GPU computing when double precision was not supported and often even single precision answers would provide slightly different results.
%
There are three key areas of accuracy that we will look at with these studies, being: Floating point precision, differences between CPU and GPU results, and IEE-754 compliance.
%

%
It was common that in early GPU computing when double floating point precision was not supported or supported well that people started thinking about GPU computing as not being accurate enough for their needs.
%
Many early attempts at GPU computing includes discussions of accuracy in order to validate the correctness of their results.
%
While modern GPGPUs support double precision much better than before making much of the worry irrelevant, it is still important to consider the accuracy of a method that runs on a new hardware and may use a new algorithm.
%

\subsubsection*{\textbf{Floating Point Accuracy} }

One of the primary concerns of the early GPU studies involved understanding the limits of floating point arithmetic on the GPU architecture.
%
Nelson in his thesis work~\cite{nelson2009monte} describes one of his primary accuracy considerations as being the difference between single and double precision calculations.
%
In older GPU hardware there was no support for double precision in the hardware and so in order to achieve double precision significantly more calculations were needed.
%
In modern GPU hardware 64 bit double precision is becoming increasingly better supported and in the GPGPU cards there are dedicated double precision units and all of the necessary hardware changes required to include them.
%

\subsubsection*{\textbf{Differences Between CPU and GPU results} }

More then the differences between single and double precision are also a concern for differences between the results that arise when using the same precision.
%
This concern can be explained by understanding how floating-point math is accomplished on a computer~\cite{goldberg1991every}.
%
There are two main reasons that differences arise.
%
The first is that floating point mathematical operations that are done in a different order might produce a different result and sue to the nature of parallel computing often you cannot know or guarantee the order a set of calculations will be preformed in.
%
The second reason is that modern day CPUs using x86 processors preform math internally on 80bit registers while a GPU does it on 32 bit (single precision) or 64 bit (double precision) registers.
%
Because of this each math operation on a CPU might stay in registers and only be rounded down to 64 bits when it is saved to memory.
%

%
Jia et. al ~\cite{jia2010development} showed that in their development of a Mont Carlo dose calculation code they could achieve speedups of 5 to 6.6 times their CPU version while maintaining within 1\% of the dosing for more then 98\% of the calculation points.
%
They considered this adequate accuracy to consider using GPUs for doing these computations.
%
Yepes et. al~\cite{yepes2010gpu} also considered accuracy in their assessment of their GPU implementation.
%
They concluded that in term of accuracy there was a good agreement between the dose distributions calculated with each version they ran, the largest discrepencies being only $\sim$3\%, and so they could run the GPU version as accurately as any general-purpose Monte Carlo program.
%
As these two groups have shown this amount of error is often very small and over the entire course of the simulation only brings 1-3\% errors.

\subsubsection*{ \textbf{IEE-754 Compliance} }
Nelson discussed accuracy in his thesis work~\cite{nelson2009monte}, stating that during the time of his work the floating-point arithmetic accuracy was not fully IEEE-754 compliant which opens the question of accuracy without a more fully featured test. 
%
Additionally, since NVIDIA as complete control over the implementation of floating point calculations on their GPUs their may be differences between generations that mitigate the usefulness of an accuracy study on one generation of hardware.
%
Current generations of the NVIDIA GPU hardware are IEE-754 compliant however. 
%
In order to address issues of floating point accuracy they have even included a detailed description of the standard and the way CUDA follows the standard showing that at least while floating point accuracy is still a concern it is no more a concern than it was on a CPU implementation.~\cite{cudaToolkitv7.5}
%

\subsection*{\textbf{Performance:}}

A second factor that is important to people making their first pass at GPU Monte Carlo is performance.
%
Most early GPU studies emphasize the speedups between CPU and GPU as the primary advantage for moving over to the GPU hardware.
%
Given the change in supercomputing designs these comparisons have become increasingly more important.
%

%
Often, performance is compared to the hardware maximmums such as peak of FLOPS or Memory Bandwidth.
%
It is often assumed that an increase in available FLOPS will translate directly into incredible performance gains.
%
In Lee et al.'s Debunking the 100X GPU vs. CPU myth~\cite{lee2010debunking}, this discussion of performance is brought into new light showing the relative performance gains for different types of applications.
%
The important thing to consider is the limiting factor between the hardware and the code.
%
Because of this comparing current performance with that of peak performance is often very misleading.
%

%
The following discussions show the relative performances of Monte Carlo transport applications that underwent their initial transformations or studies to use the GPU hardware.
%
We will not see he 100x performance that is often sought after, but instead we can understand the impact that each applications problem, algorithms, and implementation differences had on the performance as a whole.
%

\subsubsection*{\textbf{Photon Transport}}
%
Badal and Badano~\cite{badal2009accelerating} present work on photon transport in a voxelized geometry showing results around 27X over a single core CPU.
%
Their work emphasizes simply using GPUs instead of CPUs and the advanatage as GPUs continue to increase in performance faster than CPUs.
%

\subsubsection*{\textbf{Neutron Transport}}
%
Nelsons work presented in his thesis~\cite{nelson2009monte} shows a variety of models and considerations for his performance results.
%
His work solving neutron transport considered multiple models for running the problem and optimizing for the GPU.
%
The model that produced his best results shows 19.37X from a 49,152 neutrons per batch run for single precision.
%
The same model shows 23.91X when using single precision and fast math,
%
For double precision performance the model labeled model four had the fastest speedups with 11.21X and 12.66X with fast math.
%

\subsubsection*{\textbf{Gamma Ray Transport}}
%
Work presented by Tickner~\cite{tickner2010monte} on X-ray and gamma ray transport uses a slightly modified scheme from the others by launching particles on a per block basis.
%
In this way he hoped to remove the instruction level dependancies between particles running on the GPU hardware.
%
In this work he showed he was capable of producing speedups of up to 35X over a single core cpu, and a significant improvement on any per-thread methods we have seen so far.
%

\subsubsection*{\textbf{Coupled Electron Photon Transport}}
%
Jia et. al's  work~\cite{jia2010development} in a dose calculation code for coupled electron photon transport follows a relatively straight forward algorithm.
%
In their work they offload the data and computations to the GPU, simulate the particles, and then copy memory back.
%First they copy the necessary data to the GPU, then they launch with each thread independently computing the necessary work for their particle.
%%
%Finally, when the correct number of particles has been simulated the results are brought back to the CPU and the program terminates.
%%
This method produced a modest performance increase on a GPU of around 5 to 6.6X over their runs on a CPU.
%
The limitation of this speedup was attributed to the branching of the code and that effect it had on the GPU hardware.

\subsubsection*{\textbf{Track Repeating Alogorithm}}
In contrast to Jia et al's work Yepes et al~\cite{yepes2010gpu} showed that a different algorithm could greatly improve results.
%
By converting a track-repeating algorithm instead of a full Monte Carlo, Yepes et al. gained aroung 75X the performance on the GPU over the CPU.
%
It is thought that the simpler logic of this algorithm generated threads which followed closer logic to that of the algorithm presented in Jia at al's work.

\subsubsection*{\textbf{Performance Evaluation}}
Throughout all of these examples one common theme can be seen.
%
Performance can be gained doing Monte Carlo on the GPU.
%
Performance can be more difficult to get due to the highly divergent nature of the full Monte Carlo application.
%
Methods to deal with this divergence can show promising results that are worthy of further study.
%
These outcomes are expected outcomes since Monte Carlo applications are embarrassingly parallel ( good for GPUs ) but also incredibly divergent ( bad for GPUs ).
%

In this section we see a wide range in performances, from as low as 5x to as high as 75x.
%
While simplifications played a large role in the 75x algorithm we do see a full monte carlo application achieving speeds of 35x in the case of the work by Tickner~\cite{tickner2010monte}.
%
I think that is important to note that while some of the differences in performance are due to the nature of each problem being solved, the algorithmic choices made can have a significant impact on the GPU implementations.
%

\subsection*{Algorithms:}

Based on the performance studies we have just seen, it is important to highlight the algorithmic approaches that were taken so that we can understand the performances of each approach.
%
If we can clearly find algorithms that show positive performance results than other codes can implement them for potential gain.
%
In this section we are going to look closely at a few of the important or interesting algorithms we have seen attempted.
%

%
Monte Carlo transport applications tend to follow a simple model where each tracked particle is given its own thread and computations progress in an embarrassingly parallel fashion. 
%
On a GPU this also makes sense as a starting point since particles are independent and this progression leads to a simple natural parallel approach.
%
It is often pointed out however that due to the divergent nature of Monte Carlo this approach might not be the best way organize Monte Carlo codes on GPU hardware.
%

\subsubsection*{\textbf{Particle-Per-Block}}
We will first look at an alternative approach, the particle-per-block tracking algorithm described by Tickner~\cite{tickner2010monte}.
%
First each tracked particle or quantum of radiation is given to a block of threads.
%
Then calculations are performed for one particle on each block of threads.
%
For example the particle intersection tests with the background geometry can be preformed in parallel on those threads for each piece of geometry that particle might be able to collide with.
%
Areas where these parallel instructions can be utilized within a particles calculation are then used by the threads in a block computing for that particle.
%

This particle-per-block technique has shows promise as an effective way to counteract the divergence issue.
%
Particles often diverge quite quickly from one another in the code paths they follow.
%
This means that threads in a block are not always able to travel in lock step and can cause some serialization of the parallel regions.
%
By using only one particle per block the divergence problem is nearly entirely removed from the equation.
%
Additionally this method introduces new areas of parallelism that are not otherwise being taken advantage of, instruction level parallelism in the calculations for a single particle.
%

%
This method however, does not take full advantage of the parallelism in the hardware like those methods that do not mind the divergence do.
%
Many threads can execute simultaneously at once within a block and only groupings of 32 threads are held in a WARP forced into the lockstep pattern that causes potential slowdowns.
%
By running only one particle per block you are sacrificing some parallelism as not all tasks to calculate a particles path are parallel operations.
%
Additionally, since warps are scheduled out of thread blocks any particle operations that are not done in parallel among the threads of a block are serializing themselves in a similar manner as to those algorithms that run one thread per particle waiting while divergent particles have a turn.
%

In summary I think that this method has some merit if it can find enough parallel work in the thread block to execute additional parallel tasks that would otherwise be stalled if following a simpler method.
%
I also think that this method might end up showing the same characteristics of the simpler particle-per-thread model if the extra parallelism is not found, and instead loose out on the parallelism provided by particles that are not divergent from one another. 

\subsubsection*{\textbf{Event-Based Approaches}}

A second possibly more obvious method to escape the divergence issue is to switch particle tracking algorithms more dramatically from a history based version to an event based version.
%
We will have a discussion of this further in the Section on event based algorithms later in this paper.
%
Event based approaches require much more work then simply transforming an existing code to use the history based version on the GPU.
%
And as Du et al discovered in their attempt at a event based Monte Carlo version of the Archer code~\cite{xu2015archer}~\cite{du2013evaluation}~\cite{liu2015comparison}~\cite{su2013monte},  getting any speedups with that method has a whole new host of challenges to overcome.

\subsubsection*{\textbf{Voxelization Approaches}}

This method was used as for comparison on the GPUs.
%
Voxelization of a geometry was done for each voxel and this process involved: ray-stabbing number counted on the GPU and then a parity-counting method was run on the CPU to detect if the voxel was inside the mesh surface~\cite{na2010deformable}.
%
This method contained no divergence since all threads follow the exact same code paths.
%
This process is often done to voxelize geometries for before Monte Carlo codes can be run.
%
Doing this algorithm with no divergence produces a 45.5x speedup on the GPU over the CPU.
%
This example was shows in Ding et al.'s evaluation report~\cite{ding2011evaluation} in order to show the performance of the same GPU on different aspects related to Monte Carlo transport.

\subsection*{ \textbf{Evaluation:}}

\begin{table}
\caption { GPU speedup evaluation results } \label{tab:GPUPerfEval} 
\begin{center}
\begin{tabular}{ |C{.10\textwidth}|C{.10\textwidth}|C{.10\textwidth}|C{.10\textwidth}|}
\hline
 & & & \\
Case & Execution Time $T_{CPU}$ (minutes) & Execution Time $T_{GPU}$ (minutes) & Speed-up factor $T_{CPU}/T_{GPU}$ \\
 & & & \\
 \hline
 & & & \\
Neutron Transport Problem & $~ 0.496$ & $ ~0.017$ & $~ 29.2$ \\ 
 & & & \\
\hline
 & & & \\
eigenvalue/criticality problem & $ 4.25 $ & $ ~ 0.5 $ & $ ~ 8.5 $ \\
 & & & \\
\hline
 & & & \\
Voxelization & $ 2380.4 $ & $ ~ 52.3 $ & $~ 45.5 $ \\
 & & & \\
\hline
\end{tabular}
\end{center}
\end{table}

A number of studies were conducted by groups identifying the potential benefits of GPU hardware but also the hardware and software issues when developing Monte Carlo applications.
%
Among these concerns are memory limitations, lack of ECC support, lack of software optimization, limitations of SIMD architecture, clock speeds, and complex memory allocation schemes.
%
In addition the achieved performance was often not more than could be gotten with unchanged codes on a cluster.
%
In some cases though speedups were large and easy to achieve such as the 45X speedup of the voxelized approach. 
%
The results from Ding et al.'s evaluations can be seen in Table~\ref{tab:GPUPerfEval}.
%
The only strong conclusion from these works are that a clear and defined path are not yet known on how to take full advantage of the available parallelism without suffering performance penalties in turn. ~\cite{ding2011evaluation}

	\subsection{Monte Carlo and Medicine}

A look at the GPU research being done in the medical field of monte carlo transport
	\subsection{Monte Carlo and Ray Tracking}

a look at the combined efforts to put ray tracing techniques into monte carlo applications.
MC and optix.
	\subsection{Event Based Techniques}

A look at old and new event based approaches for mc applications.

\section{What is Portable Performance}

The term portable performance generally means the ability to achieve a high level of performance on a variety of architectures.
%
In this case high performance is relative to each target system~\cite{michaelwolfe2016}.
%
One important consideration then is what variety of systems are used that applications need to be portable for. 
%

%
The top ranked machines in the world currently utilize technologies like general purpose graphics processing units (GPUs, e.g., NVIDIA Tesla in Titan ), many-core co-processors (e.g., Intel Xeon Phi in Tianhe-2), and large multi-core CPUs (e.g., IBM Power, Intel Xeon in Tianhe-2 and others)~\cite{michaelwolfe2016},~\cite{top500thelist2016},~\cite{hankchilds2015}. 
%
Further, future supercomputing designs may include low-power architectures (e.g., ARM), hybrid designs (e.g., AMD APU), or experimental designs (e.g., FPGA systems)~\cite{hankchilds2015}. 
%
Given this wide array of possible architectures the value of portable performance has never before been so high.
%




\section{Monte Carlo and Portable Performance}

a look at monte carlo and pp abstractions. My alps work

\nocite{*}

% references section


\bibliographystyle{IEEEtran}
\bibliography{IEEEabrv,OralsBib}


% that's all folks
\end{document}



% An example of a floating figure using the graphicx package.
% Note that \label must occur AFTER (or within) \caption.
% For figures, \caption should occur after the \includegraphics.
% Note that IEEEtran v1.7 and later has special internal code that
% is designed to preserve the operation of \label within \caption
% even when the captionsoff option is in effect. However, because
% of issues like this, it may be the safest practice to put all your
% \label just after \caption rather than within \caption{}.
%
% Reminder: the "draftcls" or "draftclsnofoot", not "draft", class
% option should be used if it is desired that the figures are to be
% displayed while in draft mode.
%
%\begin{figure}[!t]
%\centering
%\includegraphics[width=2.5in]{myfigure}
% where an .eps filename suffix will be assumed under latex, 
% and a .pdf suffix will be assumed for pdflatex; or what has been declared
% via \DeclareGraphicsExtensions.
%\caption{Simulation results for the network.}
%\label{fig_sim}
%\end{figure}

% Note that the IEEE typically puts floats only at the top, even when this
% results in a large percentage of a column being occupied by floats.


% An example of a double column floating figure using two subfigures.
% (The subfig.sty package must be loaded for this to work.)
% The subfigure \label commands are set within each subfloat command,
% and the \label for the overall figure must come after \caption.
% \hfil is used as a separator to get equal spacing.
% Watch out that the combined width of all the subfigures on a 
% line do not exceed the text width or a line break will occur.
%
%\begin{figure*}[!t]
%\centering
%\subfloat[Case I]{\includegraphics[width=2.5in]{box}%
%\label{fig_first_case}}
%\hfil
%\subfloat[Case II]{\includegraphics[width=2.5in]{box}%
%\label{fig_second_case}}
%\caption{Simulation results for the network.}
%\label{fig_sim}
%\end{figure*}
%
% Note that often IEEE papers with subfigures do not employ subfigure
% captions (using the optional argument to \subfloat[]), but instead will
% reference/describe all of them (a), (b), etc., within the main caption.
% Be aware that for subfig.sty to generate the (a), (b), etc., subfigure
% labels, the optional argument to \subfloat must be present. If a
% subcaption is not desired, just leave its contents blank,
% e.g., \subfloat[].


% An example of a floating table. Note that, for IEEE style tables, the
% \caption command should come BEFORE the table and, given that table
% captions serve much like titles, are usually capitalized except for words
% such as a, an, and, as, at, but, by, for, in, nor, of, on, or, the, to
% and up, which are usually not capitalized unless they are the first or
% last word of the caption. Table text will default to \footnotesize as
% the IEEE normally uses this smaller font for tables.
% The \label must come after \caption as always.
%
%\begin{table}[!t]
%% increase table row spacing, adjust to taste
%\renewcommand{\arraystretch}{1.3}
% if using array.sty, it might be a good idea to tweak the value of
% \extrarowheight as needed to properly center the text within the cells
%\caption{An Example of a Table}
%\label{table_example}
%\centering
%% Some packages, such as MDW tools, offer better commands for making tables
%% than the plain LaTeX2e tabular which is used here.
%\begin{tabular}{|c||c|}
%\hline
%One & Two\\
%\hline
%Three & Four\\
%\hline
%\end{tabular}
%\end{table}


% Note that the IEEE does not put floats in the very first column
% - or typically anywhere on the first page for that matter. Also,
% in-text middle ("here") positioning is typically not used, but it
% is allowed and encouraged for Computer Society conferences (but
% not Computer Society journals). Most IEEE journals/conferences use
% top floats exclusively. 
% Note that, LaTeX2e, unlike IEEE journals/conferences, places
% footnotes above bottom floats. This can be corrected via the
% \fnbelowfloat command of the stfloats package.

