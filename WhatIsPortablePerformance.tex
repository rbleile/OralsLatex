\section{\textbf{What is Portable Performance}}

The term portable performance generally means the ability to achieve a high level of performance on a variety of architectures.
%
In this case high performance is relative to each target system~\cite{michaelwolfe2016}.
%
One important consideration then is what variety of systems are used that applications need to be portable for. 
%

%
The top ranked machines in the world currently utilize technologies like general purpose graphics processing units (GPUs, e.g., NVIDIA Tesla in Titan ), many-core co-processors (e.g., Intel Xeon Phi in Tianhe-2), and large multi-core CPUs (e.g., IBM Power, Intel Xeon in Tianhe-2 and others)~\cite{michaelwolfe2016},~\cite{top500thelist2016},~\cite{hankchilds2015}. 
%
Further, future supercomputing designs may include low-power architectures (e.g., ARM), hybrid designs (e.g., AMD APU), or experimental designs (e.g., FPGA systems)~\cite{hankchilds2015}. 
%
Given this wide array of possible architectures the value of portable performance has never before been so high.
%

\subsection{ \textbf{Abstraction Layers}}


\subsection*{\textbf{OpenMP}}

\cite{openmp}
\cite{lee2009openmp}
\cite{ayguade2010extending}

\subsection*{\textbf{OpenACC}}

\cite{wienke2012openacc}
\cite{openacc}
\cite{wang2013performance}

\subsection*{\textbf{Thrust}}

\cite{hoberock2008thrust}
\cite{hoberock2010thrust}

\subsection*{\textbf{RAJA}}

\cite{hornung2014raja}
\cite{hornung2016raja}

\subsection*{\textbf{Kokkos}}

\cite{edwards2014kokkos}
\cite{edwards2012manycore}

\subsection*{\textbf{Chapel}}

\cite{chamberlain2007parallel}
\cite{sidelnik2012performance}

\subsection*{\textbf{VTK-m}}

\cite{moreland2015vtk}
\cite{moreland2014vtk}

\subsubsection*{Dax} 
\cite{morelanddax}
\cite{moreland2011dax}
