\subsection{ \textbf{ Load Balance and Domain Decomposition} }

In order to achieve high levels of parallelism in transport problems with many geometries or zones, different parallel execution models are used.
%
The two primary models used are domain decomposition and replication.
%
Domains are sections of the problem space or geometry used for organizing where portions of a larger data set are stored. 
%
Domain decomposition involves spatial decomposition of the geometry into domains, and then the assigning of processors to work on specific domains.
%
Replication involves storing the geometry information redundantly on each processor and assigning each processor a different set of particles~\cite{procassini2005load}~\cite{o2005dynamic}.

Load balance of domain replication problems is often simply a trivial splitting of particles across processors.
%
Because of this, load balance is often discussed in conjunction with domain decomposition specifically.
%
Particles often migrate between different regions of a problem, meaning not all spatial domains will require the same amount of computational work.
%
In many applications there is at least one portion of the calculation that must be completed by all processors before all the processors can move forward with the calculation.
%
As a result, if one processor has more work than any other, all of the others must wait for that processor to complete its work~\cite{procassini2005load}~\cite{o2005dynamic}.
%
This load imbalance can cause significant issues with scalability as parallelism is increased from hundreds to millions of processors~\cite{o2013scalable}.
%

\subsection*{When to Load Balance}

%
A key consideration for performing a load balanced calculation is understanding the cost of performing that calculation.
%
If too much time is spent making sure the problem is always perfectly load balanced, then computational resources are being wasted on a non-essential calculation, resulting in overall slower performance.
%
However, if too little resources are devoted to load balancing then the problem will suffer from load imbalance and the negative effects that entails.
%
One solution is to perform load balance at the start of each cycle or iteration of a Monte Carlo transport calculation, but only when that load balance will result in a faster overall calculation~\cite{o2005dynamic}.
%
One algorithm to determine when to load balance is to use a
criterion that can be checked inexpensively each cycle to determine if a load-balance operation should take place~\cite{procassini2005load}~\cite{o2005dynamic}.
%
In these works, the first step is to compute a speedup factor by comparing current parallel efficiency ($ \varepsilon_C $) to what parallel efficiency would be if processors were to redistribute their load ($ \varepsilon_{LB} $).
%
The second step is to predict the run time by using the time to execute the previous cycle ($ \tau_{Phys} $), the speedup factor ($S$), and finally, the time to compute the load balance itself ($ \tau_{LB} $).
%
The final step is to compare the predicted run with and without load balancing to determine if the operation is worthwhile.
%
The equations describing this algorithm then are:

\begin{eqnarray}
S = \frac{\varepsilon_C}{\varepsilon_{LB}} \\
\tau^{'} = \tau_{Phys} \cdot S + \tau_{LB} \\
\tau = \tau_{Phys} \\
if\ (\tau^{'} < 0.9 \cdot \tau \ )\ DynamicLoadBalance()
\end{eqnarray}

\subsection*{Extended Domain Decomposition}

As an extension to the domain decomposition of meshes, O'Brien and Joy demonstrated an algorithm to domain decompose Constructive Solid Geometry (CSG) in a Monte Carlo transport code~\cite{o2009domain}.
%
One key difference between mesh and CSG geometries is that mesh geometries contain a description of cell connectivity, whereas cells defined through CSG do not.
%
In order to domain decompose these CSG cells, each cell was given a bounding box; since each domain is also a box, a test for if a cell belonging inside a domain becomes an axis-aligned box-box intersection test.
%

In addition to pure mesh and pure CSG problems other combinations are sometimes beneficial, such as the combination of mesh and CSG problems where there are large-scale heterogeneous and homogeneous regions.
%
In this method, a mesh region is embedded inside a CSG region allowing for the use of each in whichever region one or the other is more optimal
~\cite{greenman2009enhancements}.

\subsection*{Load Balance at Scale}

%
When load balancing massively parallel computers, 
examining the workload of every processor can affect scalability.
%
O'Brien, Brantley and Joy presented a scalable load balancing algorithm that runs in $\Theta ( log ( N ) )$ by using iterative processor-pair-wise balancing steps that will ultimately lead to a balanced workload.
%
Their algorithm was demonstrated scalability on up to two million processors on the Sequoia supercomputer at Lawrence Livermore National Laboratory~\cite{o2013scalable}.
%

%
The pair-wise load balancing scheme maintained efficiency of 95\% at 2 million processors while the not load balanced runs dropped in efficiency to around 68\% at 2 million processors.
%
In addition, the load-balanced version is able to maintain near perfect scaling up to 2 million processors.
%
By dispersing the workload over processors effectively it also decreases the overall tracking time~\cite{o2013scalable}.
%

Algorithms that interact with particles and geometries are affected when domain decomposition is added.
%
Three algorithms are described in~\cite{o2015particle} which are modified once domain decomposition is added.
%
Specifically a global particle find algorithm, a test for done, and domain neighbor replication.
%
The global particle find is the algorithm used to find where a particle is currently located in the geometry.
%
After domain decomposition a tree search was added to quickly decide which domain a particle is in before then searching in specific geometry elements.
%
The test for done algorithm, which reports if there are any particles left to process, can be easily achieved by using MPI\_I allreduce() in place of a complex hand coded algorithm.
%
Lastly, the domain neighbor replication was found to be an important algorithm to combine with domain decomposition as it increased achieved load balance and reduced the total memory usage.
%


