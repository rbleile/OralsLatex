\subsection{\textbf{Monte Carlo and Medicine}}

A perspective of Monte Carlo transport that often gets overlooked by others in the field is the area of Monte Carlo transport for medicine.
%
Radiation transport calculations are used for dose estimations in patients and require close to real time, highly accurate solutions on desktop style machines.
%
The following are descriptions from three applications of medical Monte Carlo transport followed by an evaluation of the effect GPUs have had on the field.

\subsection*{ Electromagnetic Monte Carlo transport in GMC}

Janhnke et al. ~\cite{jahnke2012gmc} in 2012 described his groups efforts to develop the code named GPU Monte Carlo (GMC).
%
GMC is a GPU implementation of the low energy electromagnetic portion of the Geant4 code using the CUDA interface.
%
GMC runs in a thread per particle style operating on 32768 particles at a time (128 blocks of 256 threads).
%
GMC runs through a series of kernel launches in a loop each handling one important aspect of the physics.

The raw performance differences between the CPU version and the GPU implementation for the problems tested is amazing for this problem.
%
The average for their study showed the GMC histories being computed at a rate of 657.60 histories every milli-second compared to the Geant4 CPU with histories computed at 0.137 histories per milli-second.
%
Comparing these two numbers produces a speedup factor of 4860 while maintaining reasonable accuracy in all cases between CPU and GPU with accuracies greater than 95\% in all regions.
%
Total runtimes were also brought down to the hundreds of seconds showing the possibility for clinical usage of applications like this.~\cite{jahnke2012gmc}

\subsection*{ Proton Therapy in gPMC}

Accurately computing radiation doses is a critical part of proton radiotherapy, and Monte Carlo simulations are considered to be the most accurate method to compute those dose calculations.
%
Given the long time required for traditional applications to use this technique, clinical application have been severely limited.
%
Jia et al.~\cite{jia2012gpu} describes a fast dose calculation code, gPMC, and how it might enable clinical usage of Monte Carlo proton dose calculations.

%
The code gPMC was developed in CUDA for use on a GPU.
%
Using a batching system to launch groups of particles from a particle stack.
%
gPMC runs for between 6 and 22 seconds to generate passing rates between 95\% and 99\%.
%
The authors except that they have successfully developed a dose calculation code under a certain set of restrictions and are hopeful that their future work will be able to meet with continuing success as they expand the context for their application.~\cite{jia2012gpu}

\subsection*{Electron-Photon Transport in DPM }

Jia et al.~\cite{jia2010development} describes the development of a CUDA based Monte Carlo coupled electron-photon application for dose planning, called DPM (dose planning method).
%
 Their scheme involves launching a kernel on the GPU that simulates all of the particle histories necissary to reach some target number of source particles.
 %
 Each thread of their kernel simulates the history of one source particle and all secondary particles that it generated.
 %
 The kernel ends with an atomic gathering of all the dosing data.
 %
 DPM was only able to achieve speedups of around 5-6.6x on the GPU over the CPU, but did get excellent agreement on relative uncertainties in their results.~\cite{jia2010development}
 
 Jia et al.~\cite{jia2011gpu} revisits their DPM code and is able to change speedups of 5-6.6x into speedups of approximately 69 - 87x.
%
DPMs main algorithm changed in a few significant ways.
%
First a single thread only computed the history of a single particle and any additional particles were placed on a stack for a future iteration.
%
Secondly the photon and electron physics was separated into different kernels so that threads would experience less divergence when handling the necessary code paths.
%
Other factors such as a better random number generator and use of the hardware linear interpolation features were also done.
%
With the additions of new features and improvements, DPM re-evaluated their accuracy to be not statistically significant in over 96\% of regions for all problems they tested.
%
Given the now excellent speedups of 69-87x and acceptable accuracy ranges, real time speeds for realistic problems was achieved.
~\cite{jia2011gpu}


\subsection*{Evaluation}

These three projects show a variety of problems in the medical Monte Carlo field.
%
They are each accomplishing their tasks on a single GPU as opposed to a cluster of CPUs.
%
There are numerous stated benefits to this, cost of purchasing and operating a cluster against purchasing a single GPU being a large factor.
%
In each case speedups were achieved that were adequate to bring the time if their simulations down to those that would be useful in a clinical environment.
%
